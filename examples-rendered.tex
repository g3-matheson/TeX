\documentclass{article}
\usepackage{kat}

\begin{document}

\begin{algorithm}
\caption{Finding pairs $(i,j)$ such that $i \equiv j \mod x$}
    \begin{algorithmic}
    \Function{ModPairs}{\texttt{int[]} nums, \texttt{int} $x$}
        \State $n \gets length(nums)$
        \For{$i \gets 1, n-1$}
            \For{$j \gets i+1, n$}
                \If{$i \, \% \, j = x$}
                    \State\texttt{print}(``Indices (\{i\},\{j\}) with values nums[i], nums[j]")
                \EndIf
            \EndFor
        \EndFor
    \EndFunction
    \end{algorithmic}
\end{algorithm}

\begin{algorithm}
\caption{Power Set $\cP\bp{\texttt{int[]}}$}
    \begin{algorithmic}[0]
    \Function{PowerSet}{$\texttt{int[]} T$}
        \State $\texttt{Queue<int[]>} \ q$							\Comment{declare queue}
        \State $q.queue(\texttt{[]})$								\Comment{start with $\emptyset$}
        \ForAll{$t \in T$} 						\Comment{$\forall \ t$, create new subsets by appending $t$ to all subsets}
            \While{$\texttt{true}$} 								\Comment{iterate through queue until $\emptyset$}
                \State \texttt{int[]} $subset \gets q.deqeue()$
                \State \texttt{int[]} $newSubset \gets subset.append(t)$ \Comment{append $t$ to subset}
                \Statex
                \State $q.queue(newSubset)$							\Comment{queue $[subset, t]$}
                \State $q.queue(subset)$							\Comment{requeue $subset$ \textbf{after}}
                \If{$subset = \texttt{[]}$} 
                    \State \texttt{break}							\Comment{stop at $\emptyset$}
                \EndIf
            \EndWhile
        \EndFor
        \State \textbf{return} $q$
    \EndFunction
    \end{algorithmic}
\end{algorithm}

\begin{algorithm}
    \caption{Proposed Critical Section Resolution}
    \begin{algorithmic}
        \State bool \texttt{flag} [2] \Comment{Initially False} 
        \State int \texttt{turn}; \Comment{Initilally 0}
        \DoWhile
            \State \texttt{flag}[i] = True \Comment{i \texttt{==} 0 for $P_0$ and 1 for $P_1$} 
            \While{\texttt{flag}[1-i]}
                \If{\texttt{turn} == j}
                    \State \texttt{flag}[i] = False
                    \While{\texttt{turn} == j} \LComment{\BoxedString[fill=forest]{Do nothing, just wait.}}
                    \EndWhile
                    \State \texttt{flag}[i] = True
                \EndIf
            \EndWhile
            \LComment{\BoxedString[fill=algored]{Critical Section Code Here}}
            \State \texttt{turn} = j;
            \State \texttt{flag}[i] = False
            \LComment{\BoxedString[fill=forest]{Rest of the Code Here}}
        \DoWhileEnd{True}
    \end{algorithmic}
\end{algorithm}

 Tree example

 \begin{tikzpicture}
[level distance = 10mm,
 level 1/.style={sibling distance=30mm},
 level 2/.style={sibling distance=15mm}]
	\node {K}
		child {node {H}
			child {node {X}}
			child {node {C}}}
		child{node {R}
			child{node {Q}
				child{node {M}
					child{node {A}
						child{node {Y}}
						child{node {S}}}
					child{node {W}}}
				child{node{D}}}
			child{node {L}}};
 \end{tikzpicture}

 Graph (and array) example
 \begin{figure}[h!] \centering
     \begin{minipage}[h!]{0.5\textwidth} \centering
         \begin{tikzpicture}[main/.style = {draw, circle}, node distance = {16mm}]
             \node[main, fill=strawberry, draw=\mypagecolor] (1) {$N$};
             \node[main] (2) [above left of=1] {$A$};
             \node[main] (3) [below left of=2] {$W$};
             \node[main] (4) [right of=1] {$D$};
             \node[main] (5) [below of=1] {$T$};
             \node[main] (6) [above right of=4] {$E$};
             \node[main] (7) [below right of=6] {$M$};
             \node[main] (8) [above right of=7] {$K$};

             \draw (1) -- node[midway, above, sloped] {8} (2);
             \draw (1) -- node[midway, above, sloped] {3} (3);
             \draw (1) -- node[midway, above, sloped] {1} (4);
             \draw (2) -- node[midway, above, sloped] {2} (3);
             \draw (3) -- node[midway, above, sloped] {5} (5);
             \draw (2) -- node[midway, above, sloped] {4} (6);
             \draw (4) -- node[midway, above, sloped] {4} (7);
             \draw (5) -- node[midway, above, sloped] {2} (7);
             \draw (6) -- node[midway, above, sloped] {3} (7);
             \draw (6) -- node[midway, above, sloped] {7} (8);
             \draw (7) -- node[midway, above, sloped] {9} (8);
         \end{tikzpicture}
     \end{minipage}
     \begin{minipage}[h!]{0.45\textwidth} \centering
     $\begin{array}{|c|c|}
         \texttt{Node} & \texttt{distance} \\
         \hline
         \texttt{D} & \infty \\
         \texttt{W} & \infty \\
         \texttt{A} & \infty \\
         \texttt{M} & \infty \\
         \texttt{T} & \infty \\
         \texttt{E} & \infty \\
         \texttt{K} & \infty
     \end{array}$
     \end{minipage} 
 \end{figure}
\end{document}